% !TeX encoding = UTF-8
% !TeX spellcheck = de_DE
% !TeX program = lualatex
% !BIB program = biber

\documentclass[british, aspectratio=169]{beamer}
\usepackage{polyglossia} \setdefaultlanguage{german}

\usepackage{fontspec} \setsansfont{Noto Sans} \setmonofont{Noto Mono}
\usepackage{unicode-math}
% \usepackage{luacode}

% \usepackage[sfdefault]{roboto}



\usepackage{../theme/beamerthemefrisbeesportverband}

\title{Frisbeesportverband Beamer Theme}
\date{\today}
\author[M. Brandt]{Matthias Brandt}
\institute{Deutscher\par Frisbeesport-Verband e.V.}

\begin{document}
\begin{frame}
  \titlepage
\end{frame}

\begin{frame} 
  \frametitle{There Is No Largest Prime Number, or is there?} 
  \framesubtitle{The proof uses \textit{reductio ad absurdum}.} 
  \begin{theorem}
    There is no largest prime number. \end{theorem} 
  \begin{enumerate} 
  \item<1-| alert@1> Suppose $p$ were the largest prime number. 
  \item<2-> Let $q$ be the product of the first $p$ numbers. 
  \item<3-> Then $q+1$ is not divisible by any of them. 
  \item<1-> But $q + 1$ is greater than $1$, thus divisible by some prime
    number not in the first $p$ numbers.
  \end{enumerate}
\end{frame}

\begin{frame}{A shorter title}
  \begin{itemize}
  \item one
  \item two
  \end{itemize}
\end{frame}


\end{document}
